\clearpage % Rozdziały zaczynamy od nowej strony.

\section{Goal}

The aim of this work is to achieve a way of running PenPoint OS on a modern
hardware with performance that would make possible to use in a comfortable
manner.  It should be possible to use a mouse, a graphics tablet or a touch
screen for pointing.  This work could be then further used for purposes of
digital preservation of the PenPoint Operating System.

%\subsection{Prior Art}

\clearpage

\section{Prior Art}

% - not popular, not much activity, forgotten system
% - strategies: QEMU, DOSBox, VirtualBox
% - difficult to do - special hardware, documentation lost (of hardware)
% - only ___ Schubiger
% - remaking of the Schubiger solution

% - forgotten system
% - no interest
% - only schubiger
% -
% - difficulty with PPOS
% - multiple versions
% - hardware
% - lost documentation
% -
% - strategies
% - rewrite
% - compatibility layer (wine)
% - emulator
% - VB, QEMU
% -
% - schubiger solution
% -
% - remaking of schubiger solution

Unfortunately there is not much interest in the revival oft the PenPoint
operating system.  Sadly, it is pretty much a forgotten system.  The only other
effort I came across that was dedicated to making PenPoint accessible and
usable again was the project of professor Simon Schubiger.

What makes this task difficult is how many different version of PenPoint there
were.  PenPoint OS was released for a variety of computers, and most of them
had some special, rather unique components.  To make it worse, a lot of
information about that specific hardware is probably lost to time; what is more,
so are the copies of PenPoint installation media for many of these devices.

There are many possible strategies when it comes to making software accessible
for a platform that this software was not intended for originally.  Sometimes
it is possible to modify parts of the source code to port it or even rewrite
the whole thing from scratch if preferable and necessary.  This is almost
always not possible and not feasible.  When the source and destination
platforms are similar enough, some sort of compatibility layer can be used.
Wine works this way to allow running Windows applications on Unix-like systems.
However the most popular and versatile tool for this task is an emulator.
There is a great deal of variety in the available emulators, many of them aim
to solve different, sometimes contradictory problems.  The two most popular
open-source and libre emulators are QEMU and VirtualBox.  The goal of the
former is performance, the goal of the latter is compatibility and best
possible experience out of the box.

Professor Simon Schubiger chose the JPC emulator.  The JPC emulator is
a relatively simple emulator, especially compared to such programs as QEMU, and
it is written in Java.  These features made it easy to create a runnable proof
of concept implementation that allowed to access PenPoint OS.  Unfortunately, it
was not good enough for actual usage.  It was, however, a starting point.  In
the following sections I will describe in greater detail what JPC is, the
modifications created by professor Simon Schubiger, and the attempt to recreate
the running PenPoint solution and port it to QEMU.
